\documentclass[a4paper, 11pt]{book} % A4 paper size and default 11pt font size

\newcommand*{\plogo}{\fbox{$\mathcal{PL}$}} % Generic dummy publisher logo

\usepackage[utf8]{inputenc} % Required for inputting international characters
\usepackage[T1]{fontenc} % Output font encoding for international characters
\usepackage{stix} % Use the STIX fonts

\begin{document}
\begin{titlepage} % Suppresses displaying the page number on the title page and the subsequent page counts as page 1
	
	\raggedleft % Right align the title page
	
	\rule{1pt}{\textheight} % Vertical line
	\hspace{0.05\textwidth} % Whitespace between the vertical line and title page text
	\parbox[b]{0.75\textwidth}{ % Paragraph box for holding the title page text, adjust the width to move the title page left or right on the page
		
		{\Huge\bfseries A Collection of \\[0.5\baselineskip] \LaTeX ~Templates}\\[2\baselineskip] % Title
		{\large\textit{A predictable subtitle}}\\[4\baselineskip] % Subtitle or further description
		{\Large\textsc{gordon freeman}} % Author name, lower case for consistent small caps
		
		\vspace{0.5\textheight} % Whitespace between the title block and the publisher
		
		{\noindent The Publisher~~\plogo}\\[\baselineskip] % Publisher and logo
	}

\end{titlepage}
sie sind leise sauber und gut für das

klima elektroautos dienen soll die

zukunft gehören elektromobilitäts dass

es die chance unabhängig vom öl zu

werden

es geht um viel um arbeitsplätze und die

zukunft der automobilbranche in

deutschland müssen das im eigenen land

vor machen

deutschlands autobauer

sie fahren weltweit vorne weg und sehen

sich auch beim elektroauto auf der

überholspur

zehn verschiedenen fahrzeugmodellen

angebot wer hat das begehen neue wege zu

neuer stärke

doch wie weit sind wir bei den

elektroautos wirklich deutschland ist

das schlusslicht in der national in der

elektromobilität wie erfahren hinterher

versagen politik und industrie beim

wechsel zu der neuen technik findet die

auto zukunft vielleicht sogar ohne uns

statt

wir sind in münchen ein brandneuer

elektro smart wird mit den logos des

arbeitersamariterbundes kurz asb beklebt

der wohlfahrtsverband will flagge zeigen

sein pflegedienst soll mit ökostrom und

ohne abgase fahren gut für das grüne

gewissen ulrich lindl vom münchner asb

ist von dem bislang meistverkauften

elektrofahrzeug in deutschland überzeugt

obwohl es deutlich teurer ist als ein

normales auto ich hab's durchgerechnete

verständlich es in den ev seinen

vereinsmitgliedern rechenschaft schuldig

wir müssen schon auf den euro schauen

und ich habe noch sechs jahren so lange

wollen wir die autos betreiben

mindestens nach sechs jahren immer noch

eine unterdeckung von 2000 euro

errechnet diese höheren kosten nahm der

asb aber wegen der guten sache in kauf

und bestellte bei daimler acht

elektroautos

doch dann die überraschung der konzern

konnte nur drei liefern

grund war das schlicht nur drei

verfügbaren fünf waren nichts aus der

verbrauch dann bekommen sie die

restlichen wir hoffen dass dass anfang

2016 muss wissen es nicht genau

was ist da los wir fragen mehrmals nach

bei daimler gibt es lieferprobleme beim

elektro smart schließlich die

entlarvende antwort daimler hat die

fertigung der aktuellen version des e

autos eingestellt neue elektro modelle

werden erst in der zweiten jahreshälfte

2016 vorgestellt

die kunden müssen daher wohl noch lange

auf neue elektro autos warten

doch die frage ist warum werden sie

vorläufig nicht mehr produziert daimler

gibt zu der konzern benötigt die montage

bänder um normale also benzin betriebene

smarts für die wichtigen überseemärkten

china und usa zu fertigen

das stuttgarter unternehmen liefert also

lieber fahrzeuge mit verbrennungsmotoren

ins ausland als elektro autos zu

verkaufen

gleichzeitig verkündet vorstandschef

dieter zetsche in aller öffentlichkeit

bei elektroautos weltweit führend zu

sein

die aufgabe dort peter scriba

weltmarktführer auf der angebotsseite zu

werden

das haben wir erreicht weltweit

führender anbieter und dann kann der

konzern nicht mehr liefern

wie passt das zusammen welche rolle

spielt elektromobilität tatsächlich für

die deutsche industrie auto tausch mit

dieser ungewöhnlichen aktion will das

land berlin menschen für elektro autos

begeistern

sie macht mit amira bodo die junge

berlinerin muss den schlüssel ihres

benziners abgeben

dafür kann sie jetzt zehn tage lang

kostenlos ein elektrofahrzeug testen

sehr kompliziert das denke ich werde ich

jetzt in den nächsten zehn tagen

herausführen

es geht sofort los die erste fahrt führt

quer durch berlin

amira bodo wird das auto vor allem in

der stadt nutzen

ihr erster eindruck es wäre doch ganz

sanft also ein leichtes tippen genügt

und der wagen los

klingt überzeugend aber trotzdem kaufen

will so ein fahrzeug in deutschland kaum

einer im vergangenen jahr brachten die

deutschen autobauer offiziell nur 8463

stromer auf die straße doch selbst diese

zahl ist geschönt

denn die hälfte davon haben sie auf sich

selbst zugelassen

lediglich 4814 elektroautos wurden

tatsächlich an kunden verkauft

dabei sind bereits fast 20 modelle auf

dem markt

reine elektroautos und so genannte plug

in hybride wie dieser porsche cayenne

er hat einen normalen benzinmotor und

kann zusätzlich kurze strecken mit strom

aus der steckdose fahren

er gilt deshalb auch als elektroauto

wir haben wegweisende innovation

etabliert insbesondere beim plug in

hybrid ist unser konzern

technologieführer die

automobilhersteller hier in deutschland

dem land der ingenieure und innovationen

haben geliefert

mitte juni in berlin die mächtigen

autobosse treffen sich mit der regierung

seit fünf jahren sind sie sich einig

deutschland soll vorreiter bei der

elektromobilität werden

man kennt sich die atmosphäre ist gelöst

dabei müsste doch hier krisenstimmung

herrschen denn die bisherige bilanz ist

eine katastrophe insgesamt rollen auf

deutschen straßen derzeit nur 32.000

elektroautos

dabei sollten es bald eine million sein

das ist mittlerweile eine

märchengeschichte unmöglich zu erreichen

die kanzlerin kommt sie trifft auf die

chefs von konzernen die

milliardengewinne einfahren

trotzdem haben die bosse nur eines im

sinn geld vom staat

jetzt ist die politik am zug

sie hat es selbst in der hand wie

schnell e mobilität vorankommt hinter

verschlossenen türen machen die

konzernchefs dann druck das bekommt die

kanzlerin zu spüren

ich habe aus der heutigen veranstaltung

gelernt man erwartet von uns noch in

diesem jahr eine antwort

doch will sie diese erwartung erfüllen

und steuergelder für elektroautos locker

machen

das ganze ist nicht verkündet zu reden

für die autobranche ist deshalb klar

schuld ist allein die regierung wenn

deutschland bei der elektromobilität

nicht vorankommt doch ist das so einfach

wir recherchieren ist die deutsche

industrie bei elektroautos wirklich

weltweit führend sowie die konzernbosse

das behaupten

beispiel usa der wichtigste markt

schlechthin für elektroautos 123.000 die

fahrzeuge wurde im vergangenen jahr dort

verkauft die amerikanischen hersteller

sicherten sich fast die hälfte des

marktes

es folgen die japaner die angeblich

weltweit führenden anbieter aus

deutschland überzeugten nur neun prozent

ein paar moves zeugnis auch in europa

fahren die deutschen hersteller nicht

vorneweg auf platz eins liege die

japaner mit großem abstand dahinter die

deutschen

dicht gefolgt von den französischen

herstellern von wegen

vorreiter liegt das vielleicht an den

fahrzeugen

wir treffen die auto testerin an ihrer

abo du nach der arbeit

sie hat strom gezapft die batterie ihres

e autos ist jetzt voll

laut hersteller soll eine ladung für 190

km reichen

bei ihr werden aber jetzt nur 122 km

angezeigt und ohne klimatisierung und da

habe ich auch zwei weitere kilometer

also 124 kilometer das schaffe ich auch

locker man an einem tag ab zu fahren

auch wenn ich in der stadt wohne

also es schon ja ein bisschen

deprimierend muss ich sagen ich war

begeistert aber das bisschen wenig und

dann ist dieses auto auch noch teuer

35.000 euro kostet es fast 12.000 euro

also rund ein drittel mehr als ein

vergleichbarer benziner im vergleich zu

anderen autos viel zu überteuert

warum kostet der wagen so viele

fahrzeuge von französischen oder

japanischen herstellern sind deutlich

günstiger

wir fahren nach leipzig hier montiert

bmw in einem nagelneuen werk den i3

die karosserie wird aus karbon gefertigt

dieses high tech material ist derzeit

noch teuer aber es wird nur geklebt in

wenigen arbeitsschritten wenn das super

harte carbon ist extrem leicht

daher lassen sich daraus große teile

fertigen die produktion der dadurch

wesentlich effektiver und billiger und

auch das werk selbst hat wenig gekostet

nur 400 millionen euro es ist

vergleichbar wenig weil wie gesagt ein

klassisches presswerk entfällt eine

klassische lackierer einfällt ist nicht

nur die haupt energie treiber sondern

auch die haupt investitions treiber in

einer produktionsstätte und damit haben

natürlich einen doppelten vorteil was

das thema nachhaltigkeit anbelangt

sowohl ökologisch ökonomisch das werk

benötigt nur halb so viel energie wie

eine normale autofabrik hinzu kommt weil

das fahrzeug so leicht ist kann der

konzern eine kleinere batterie einbauen

auch das spart kosten wenn die

produktion der abt künstlich ist dann

stellt sich erst recht die frage warum

ist der i3 so teuer dass vollkommen

nachhaltig ist mit dem tiefsten

antriebsstrang

wer das fahren will der bekommt es bei

uns zu diesen preisen

es ist offenbar gewollt den preis hoch

zu halten

die autos sollen keine schnäppchen

werden die folge ist

im vergangenen jahr hatte bmw weltweit

nur 16.000 52 3 verkauft das waren nicht

einmal ein prozent der ausgelieferten

fahrzeuge des konzerns

offizielles bmw zufrieden wir wollen

wissen könnte der autobauer überhaupt

auf die schnelle mehr elektrofahrzeuge

produzieren behalten wir normalerweise

stillschweigen

wir fragen den autoexperten stefan

bratzel

welche bedeutung haben elektroautos für

die konzerne wollen sie überhaupt schon

viele autos verkaufen

je mehr sie im bereich der

elektromobilität voran gehen desto eher

wird ihre kompetenz im benzin dieser

bereich entwertet also es ist im moment

auch nicht so richtig rational für die

hersteller jetzt ganz weit nach vorne zu

springen

wir sind bei familie müller auf dem land

in brandenburg auch constanze müller

macht beim berliner autor tausch mit

sie kann 14 tage lang einen e golf

testen die mutter dreier kinder ist

logopäden und kommt ohne auto nicht aus

damit die kinder nicht schon um kurz

nach sechs mit dem bus los müssen bringt

sie die beiden großen selbst zur schule

wie finden die das elektroauto

das elektro fahrgefühl hat auch

constanze müller schnell überzeugt da

ist schon groß

es hat mich auch schon erwischt wäre gar

nur nicht der preis auch ihr

testfahrzeug kostet 35.000 euro und

damit 15.000 mehr als der benziner

würden sie so viel geld in die hand

nehmen um sich so ein auto zu kaufen

nein

35.000 euro dabei sieht ein e golf aus

wie ein stinknormaler golf fahrer des

elektro modells können nicht einmal

zeigen dass sie ökologisch sauber

unterwegs sind

warum bekommen sie kein eigenes design

und den verkauf zu fördern

wir recherchieren und finden eine

erklärung und zwar vom vw chef

höchstpersönlich bei uns laufen

elektroautos in hybride nicht in

separaten bergen verband zur stoßstange

stoßstange mit den klassischen antreten

das spart viel geld und macht uns vor

allem hoch beweglich denn derzeit ist

offen wohin genau die reise bei

antrieben geht

moment mal bei den antrieben ist alles

offen glaubt volkswagen gar nicht an das

elektroauto

tatsächlich laufen in vw-werken alle

modelle vom gleichen band mal ein plug

in hybrid oder ein benziner mal ein

elektromodell die karosserie aus blech

und stahl ist immer gleich nur der

antrieb ist anders das spart immens

kosten

und noch wichtiger der konzern kann

einfach fertigen was bestellt wird und

ist nicht festgelegt auf elektroautos

nun sie verdienen zu zeigen im moment

sehr viel geld im wesentlichen mit ihren

höherwertigen fahrzeugen mit suvs mit

ihren oberklassemodellen dort wird das

geld verdienen und diese modelle sind im

wesentlichen benzin oder diesel modelle

doch die flexible strategie des vw

konzerns hat einen nachteil

so liegt ein e golf 350 kilogramm mehr

als ein normaler golf

denn er trägt die gleiche karosserie und

zusätzlich noch schwere akkus ein

optimales elektroauto ist das daher

nicht am liebsten würden die deutschen

autobauer in der zukunft natürlich

ausschließlich verbrennungs fahrzeuge

bauen des beherrschen es ist 99 prozent

vom geschäft das läuft auch wie kommt

constanze müller mit ihrem auto zurecht

die logopädin fährt vormittags zu

patienten sie macht hausbesuche pro tag

legt sie rund 120 kilometer zurück dafür

reicht der akku doch etwas

unvorhergesehenes darf dann nicht mehr

passieren

wir hatten zum beispiel letzte woche die

situation dass ich von der arbeit kam

und genau in dem moment unser sohn von

der schaukel fiel und sich den kopf

aufschlug

mit einer großen platzwunde ganz

dringend schnell ins krankenhaus

gefahren werden musste

und das hätte ich weil das krankenhaus

knapp 30 kilometer entfernt ist mit dem

auto dann nicht mehr geschafft und wie

ist das beim auto tausch in der stadt

auch am ihrer abo duo hatte mit der

geringen leistung der batterie schon

probleme damit elektroautos wirklich

alltagstauglich werden hilft daher nur

eins die batterien müssen wesentlich

besser werden

was leistet hier die deutsche

autoindustrie

wir sind in kamenz in der nähe von

dresden

hier betreibt daimler deutschlands

einzige fabrik für batteriezellen drehen

lässt man uns nicht wird hinter diesen

toren an der technik der zukunft

gearbeitet

keineswegs daimler macht die firma zum

jahresende dicht

es bleibt dabei dass in deutschland

wirtschaftlich wettbewerbsfähig fertigen

haben wir wieder eingestellt das

eingeständnis eines scheiterns dabei hat

der staat seit 2009 rund 30 millionen

euro in die zellfertigung rund um lead

weggesteckt

ausgezahlt hat sich das nicht

nicht böse aber die entwicklung dort

enttäuscht mich jedenfalls sehr wir

wollten damals die batterieproduktion am

ende zurück nach deutschland holen

gleich neben li-tec die firma deutsche

accumotive die ebenfalls daimler gehört

hier werden aus den zellen von leadtek

komplette batterien gefertigt

die zellen sind das herzstück der akkus

in ihnen steckt ein viertel der

wertschöpfung eines elektroautos wenn li

tec zum jahresende zu spart liefert hier

ein konzern aus korea die zellen dann

ist die deutsche autoindustrie endgültig

abhängig von lieferanten aus asien und

den usa schon heute sieht es so aus

in den e autos von bmw stecken

batteriezellen von samsung drin für vw

liefert panasonic die zellen und

mercedes verbaut im elektro modell der b

klasse sogar eine komplette batterie vom

amerikanischen hersteller tesla und

nicht nur das beim bisher einzigen

reinen e auto der marke mercedes stammt

nicht nur der akku sondern auch noch der

antriebsstrang von tesla mercedes

braucht entwicklungshilfe vom

elektroauto pionier aus den usa

offensichtlich und daran will das

unternehmen auch in zukunft nichts

ändern

werden auch zukünftig zusammenarbeiten

vor allem natürlich bei der b-klasse

electric drive der antriebsstrang von

tesla stand das darf nicht der anspruch

der deutschen autoindustrie seien die

als sozusagen eine

technologieführerschaft gerade auch beim

thema antrieb einklagten einfordert also

man muss es zumindest in der nächsten

oder spätestens übernächsten batterie

generation hinbekommen dass dieser

wertschöpfungsanteil wiederum in

deutschland eben aufgebaut wird können

forschung und industrie in deutschland

das schaffen völlig neue leistungsfähige

batterien entwickeln und dann auch

produzieren wo stehen wir hier bei

dieser wichtigen technik der zukunft ist

es doch insgesamt recht wichtig auch

dass wir zwischen den automobilbauern

und denen die sozusagen die

elektrochemie von beginn an verstehen

eine solche nationale plattform haben um

uns hier über auch klare eine klare

meinungsbildung möglich zu machen

während deutschland noch um eine meinung

ringt und gesprächskreise gründet legt

der amerikanische autobauer tesla ein

rasantes tempo vor

zusammen mit panasonic baut firmenchef

elon musk in den usa gerade eine

gigantische fabrik zur batteriefertigung

auf für 6 milliarden dollar

die sogenannte giga factory schauen sie

einfach mal nach japan

panasonic city oder jetzt nach usa zur

tesla zur neuen giga factory dort wird

die akku zukunft stattfinden

definitiv im die derzeitigen stand der

dinge nicht bei uns da müsste massiv

investiert werden massiv geld in

forschung und entwicklung gegeben werden

wenn das nicht der fall ist haben wir

hier keine chance mehr ich denke das

thema ist durch

zurück in berlin beim auto tauscht wer

ein e auto fährt muss ständig strom

nachladen na mirapodo hält deshalb nach

einer ladesäule ausschau gerade ist die

in berlin kreuzberg unterwegs in einem

café will sie einen freund treffen

der wartet schon auf sie

ok

gleichauf liegen

gibt es eine tankmöglichkeit in der nähe

auf die schnelle findet sie keine

sie sucht im navi die nächste säule ist

1,2 kilometer entfernt ziemlich weit

na ja jetzt nicht so wie ich mir

vorgestellt habe das wäre dann blöd da

müsste ich wieder zurücklaufen glaube

ich lade das auto dann doch erst wenn

ich zu hause bin

also lieber gleich ins café als noch

eine viertel stunde fußmarsch einzulegen

nur ein beispiel in amsterdam wäre ihr

das nicht passiert dort stehen

stromtankstellen an jeder ecke

so hatte das kleine holland 2014 bereits

3700 öffentliche ladestationen

verglichen mit 2400 im großen

deutschland im offiziellen bericht von

regierung und industrie sind es aber

doppelt so viele 4800 ein rechentrick in

deutschland zählen sie ladepunkte je

zwei ladepunkte ergeben eine ladestation

und in allen anderen ländern werden die

ladestationen gezählt

bei constanze müller auf dem land etwa

gibt es überhaupt keine öffentliche

ladesäulen ihr wagen hing nach der

arbeit kurz am kabel

sie will jetzt nach berlin fahren um

ihre schwester zu besuch

reicht der akku jetzt dafür aus um auch

wieder zurück zu kommen braucht sie in

berlin eine schnellladestation bei der

kann sie in einer halben stunde voll

tanken statt wie sonst in acht stunden

doch wo steht so eine ladesäule

vielleicht hilft eine handy-app

über den gerichten laden sechs

ladestation uhr an denen schnell laden

könnte in ganz berlin

ja nicht viel

nur gut eine handvoll schnell ladesäulen

und dass in der millionenstadt berlin

davon steht eine einzige in vertretbarer

nähe zu ihrer schwester die steuert

constanze müller jetzt an

glück gehabt der platz ist frei

jetzt muss sie sich anmelden

dann sollte das auto in einer halben

stunde wieder volle reichweite haben

besteht bestzeit eine stunde 20 minuten

versprochen war eigentlich eine halbe

stunde das ist deutlich länger fast

dreimal so lang ich nicht mehr komisch

für schnellladestationen zum flotten

laden sind rar und sie hat eine erwischt

die nicht wirklich schnell laden kann

wie sieht das in anderen ländern aus

in england ist der schnell ladenetz gut

ausgebaut

mit turbo stromtankstellen übersät ist

auch holland und in deutschland weite

teile brachland

so findet sich auf der strecke von

hamburg nach berlin keine einzige

station zum schnelladen das schnell die

infrastruktur in deutschland komme als

katastrophal bezeichnen weil einfach die

linien die damit bestückt werden hätten

sollen allein die ihr aussage recht

hübsch schon nicht gemacht wurde

es ist einfach nichts vorhanden oder nur

sporadisch er verspricht das zu ändern

bundesverkehrsminister alexander

dobrindt wir wissen dass wir etwas zu

erledigen haben was die

ladeinfrastruktur mit betrifft wir

werden in den nächsten drei jahren 400

autobahnraststätten in deutschland mit

elektro ladesäulen ausstatten

ist das dann der durchbruch erreichen

400 tankstellen bei weitem nicht aus das

ist ein tropfen auf den heißen stein

man könnte sagen es ist besser als

nichts aber vielmehr ist es auch nicht

elektromobilität in deutschland

scheitert daran dass erstens wenig geld

in die hand genommen wird und das was

man in die hand nimmt das verzettelt man

wir recherchieren

wofür gibt die bundesregierung

eigentlich geld aus um die

elektromobilität voranzubringen

in einer datenbank finden wir hunderte

projekte der deutschen autobauer

die der steuerzahler finanziert beispiel

vw

der wolfsburger konzern hat unter

anderem fast drei millionen euro vom

staat kassiert um bis 2016 einen

plug-in-hybrid im oberen fahrzeugsegment

zu entwickeln

seltsam

der konzern hat doch gerade einen

solchen plug in hybrid vorgestellt

im passat gte wieso braucht volkswagen

dann noch hilfe vom staat

zumal der konzern im vergangenen jahr

ein rekordergebnis von 12,7 milliarden

euro verkündet hat auch bmw hat

milliarden euro gewinn gemacht aber für

diese schicke schnell ladesäule vor der

bmw welt in münchen musste der

steuerzahler aufkommen

und zwar mit fast 900.000 euro

und daimler bekam 1,8 millionen euro für

ein projekt damit die eigenen

konzernmitarbeiter harry fahrzeug zur

arbeit pendeln können

dafür wurden 260 stromer angeschafft

natürlich aus dem hause daimler

insgesamt zeigt sich die autokonzerne

kassieren wesentlich mehr

seit es staatliche gelder für

elektroautos gibt so hat zwischen 2001

und 2007 volkswagen 13 3 millionen euro

erhalten

in den sieben jahren danach war es dann

fast neunmal so viel bei bmw stieg die

förderung auf das fünffache an

bei daimler verdreifachten sich die

subventionen vom staat

konkrete projekte zu bezuschussen von

automobilherstellern ist aus meiner

sicht nicht notwendig und auch nicht

zielführend die automobilhersteller

werden sozusagen an dem thema immer dann

arbeiten wenn sie sozusagen eine

zukunfts chance darin sehen und um das

zu tun ist genug forschungsgeld für auch

bei den unternehmen zur verfügung da

braucht es wenig forschungsförderung von

öffentlichen stellen muss der

steuerzahler denn überhaupt geld in die

hand nehmen und elektroautos auf die

straße zu bringen

nein das muss er nicht eigentlich

reichen nämlich gesetzliche vorgaben und

die gibt es bereits

denn schon heute müssen die autobauer

co2 ziele aus brüssel erfüllen

im schnitt darf die flotte eines

herstellers derzeit 130 gramm co2 pro

kilometer ausstoßen

das schafft die industrie obwohl sie

immer mehr spritschluckende geländewagen

verkauft

doch 2020 sieht das ganz anders aus

dann müssen die autobauer 95 gramm pro

kilometer erreichen und zwar für die

gesamte verkaufte flotte dafür brauchen

sie die elektroautos und zwar viele

wieder industrieberater thomas güttler

errechnet hat innerhalb von sieben

jahren

was würde geschehen wenn das nicht

gelingt weiß es nicht gelingen sollte

und die deutschen hersteller die co2

vorgaben reisen sollten bedeutet dass

ein empfindlicher image verlust bei

gleichzeitigen strafzahlung in

milliardenhöhe 20 elektroautos wie soll

das gehen

wir sind in kitzbühel hier stellt

mercedes den neuen gl geländewagen vor

solche schweren autos sollen die co2

bilanz retten ist das wieder ein märchen

tatsächlich gibt es von dem modell einen

sogenannten plug in hybrid gehen kann

die autopresse hier testen

wir begleiten frank mertens

chefredakteur eines automagazins der

wagen hat einen benzinmotor mit 330 ps

an bord und zusätzlich einen

elektroantrieb mit batterie wie man an

der steckdose aufladen kann

damit kommt das fahrzeug 31 kilometer

weit

fast zweieinhalb tonnen

und das ist natürlich ein trumpf von

auto und dann habe ich hier die option

halt elektrisch unterwegs zu sein und

das gibt ein gutes gefühl und auch

entsprechend geht das so ein spaßfaktor

doch wie ist das mit dem verbrauch bei

unserem test im modus hybrid greift der

motor zunächst auf die batterie zu nach

31 kilometern ist die dann leer

wir fahren weiter immer im hybridmodus

und nur landstraße insgesamt rund 60

kilometer sonderlich effizient unterwegs

durchaus ein bisschen ein bisschen

flotter wir fahren jetzt 90 km h auf der

landstraße und da sind jetzt 83 liter

laut anzeige

okay für ein auto dieses dieser größe

dieses segmentes da kann man auch nicht

wirklich mehr

doch moment 8 3 liter mehr als doppelt

so viel wie offiziell angegeben

laut normverbrauch soll der schwere

wagen im schnitt nur 3 3 liter schlucken

und damit lediglich 78 gramm co2 pro

kilometer ausstoßen so viel wie ein

kleinwagen

gut für die industrie denn nur dieser

offizielle abgas wert ist für autobauer

von bedeutung

er drückt die co2 flotten bilanz nach

unten

so wird die umweltbilanz geschönt

auch bmw legt seine fahrzeugflotte an

die leine

der neue x5 etwa sein offizieller co2

wert als plug-in-hybrid 77 gramm pro

kilometer

schwere autos mit gleich zwei motoren

sollen also die umweltbilanz retten

am ende des tages wird jeder hersteller

für diese ziele die co2 ziele des jahres

2020 so viele elektrofahrzeuge in sein

seiner flotte installieren so dass es

gerade zum erreichen der co2 grenzwerte

reicht und so günstig wird man dann die

fahrzeuge machen das eben genug

fahrzeuge auch verkauft werden das wird

sicherlich die rechnung seien auch audi

setzt auf plug in hybride alexander götz

fährt einen a3 e tron mit benzin und

elektroantrieb im rahmen eines

pilotprojektes offiziell kommt der wagen

sogar auf einen wesentlich geringeren

co2 wert als ein kleinwagen nur 35 gramm

pro kilometer

und er schluckt angeblich nur 1,5 liter

wie realistisch ist das dieses fahrzeug

lässt sich im schnitt mit dreieinhalb

bis vier litern pro 100 kilometer

bringen also in der praxis mehr als

doppelt so viel wie von audi angegeben

wie kommen die sagenhaft niedrigen

offiziellen werte zustande die fast alle

deutschen autobauer angeben

die lösung sehen wir im adac

technikzentrum in landsberg

hier misst der automobilclub wie viel

co2 fahrzeuge ausstoßen

alex knöfel kennt deshalb die tricks mit

denen autobauer die offiziellen

messungen schön da ist es durchaus

praxis dass man mal den rechten

außenspiegel ab nimmt um ein bisschen

weniger luftwiderstand zu haben da ist

es praxis dass man diese spaltmaße mit

tape mit klebeband maskiert um als

fahrzeug windschlüpfriger zu kriegen und

im extremfall werden sogar radkästen

nochmal abgeklebt nur damit das auto ein

paar meter länge rollt auch bei der

batterie wird getrickst

die kann vor dem test voll aufgeladen

sein und darf dann aggregate antreiben

die normalerweise der motor speist

erlaubt ist teilweise auch das fahrzeug

leichter zu rechnen als es eigentlich

ist

in der summe wenn man alle löcher

tatsächlich optimal ausnutzt und

trotzdem noch die richtlinie ein held

sind sicher 15 bis 20 prozent

und das ist doch eine beachtliche größe

wohlgemerkt diese tricksereien sind bei

allen messungen völlig legal bei plug in

hybriden kommt noch etwas besonderes

hinzu eine spezielle rechenformel die so

tut als ob mit dem verbrennungsmotor

immer nur kurze strecken gefahren würden

die formel aus der typ genehmigungs

richtlinie setzt voraus dass ich mit

meinem plug in hybriden alle 25

kilometern an die steckdose gehe und das

sieht in der praktisch ganz anders aus

die leute haben doch ein anderes

nutzungsverhalten mit den autos und sind

dann doch enttäuscht wenn sich der wert

der so schön im prospekt aussah in der

praxis nicht zeigt eigentlich sollte es

damit bald vorbei sein denn für die

flotten bilanzen sollen ab 2020 weltweit

neue regeln zur abgas messung gelten wie

die in der eu umgesetzt werden darüber

ringen politik und industrie derzeit

heftig hinter den kulissen uns liegen

zahlreiche dokumente vor die zeigen die

autolobby nutztiere ganze macht um

strengere messverfahren zu verhindern

das wird ihr leicht gemacht denn die eu

kommission lädt die lobby vertreter

regelmäßig zu sitzungen ein

dort erhebt vor allem der europäischen

autoverbands acea dann forderungen was

damit passiert zeigt dieses dokument

die eu folgt sich mal den vorschlägen

des ac e a

oder berücksichtigt dessen bedenken

der ac e a es ist der europäische

lobbyverband von 15 auto und lkw

herstellern vorne mit dabei

die deutschen

auch diese e-mail einiger konzerne an

die kommission enthüllt den massiven

einfluss

dort heißt es konkret der ac e a wird

als höchstgeschwindigkeit beim

verbrauchs test nur 145 stundenkilometer

akzeptieren

und tatsächlich der gesetzentwurf

übernimmt genau diesen wert mit dem

entlarvenden hinweis eines beamten

die kommission würde eine

höchstgeschwindigkeit von 160

stundenkilometern bevorzugen aber als

kompromiss mit dem ac e a

scheint 145 stundenkilometer akzeptabel

wir sind in brüssel

beim ringen um den übergang zum neuen

abgastest vertritt greg archer

europäische umweltgruppen gegen die auto

lobby steht da meist ganz allein

oft sind es 30 leute von der

autoindustrie und ich aber ich kann

schon für meine sache kämpfen

wir wollen von ihm erfahren was konkret

versuchen die autohersteller

durchzusetzen

die autoindustrie will beim übergang vom

jetzigen co2 ziel zu den neuen co2

zielen unter den neuen abgastests

erreichen dass sie alle jetzt

existierenden schlupflöcher aus nutzen

darf im kern sagt sie sind diese

manipulationen sind uns heute erlaubt

sie sollten uns auch unter den neuen

tests angerechnet wird

so hat der europäischen autoverbands

acea für eine sitzung zur co2 messung am

siebten mal detailliert sechs

rechentricks aufgelistet auf die er im

zusammenhang mit den tests weiter

bestehen und die autolobby kämpft nicht

allein die deutsche regierung hilft

schlupflöcher zu erhalten

wir konfrontieren die

bundesumweltministerin mit diesem

vorwurf der umweltverbände

jetzt haben umweltverbände sich an sie

gewendet und haben gesagt die

bundesregierung unterstützt die

autoindustrie

diese schlupflöcher zu erhalten was

sagen sie dazu

nein das stimmt nicht wir sind sehr

daran interessiert zusammen mit der

europäischen kommission tatsächlich

berechnungsgrundlagen zu haben die auch

tatsächlich stimmen denn es hilft ja

nichts wenn wir uns schön rechnen damit

gewinnen wir kein vertrauen ist doch

greg archer in brüssel zeigt uns eine

vorlage für die sitzung im mai in der

die deutsche regierung für dieselben

sechs test bereiche wie der ac e a

fordert schlupflöcher für die

autoindustrie weiter zuzulassen

die position der deutschen regierung ist

genau die gleiche oder sehr ähnlich wie

die der autoindustrie politik und

industrie hebeln also gemeinsam die co2

vorgaben aus brüssel aus sie zeigen

damit unverblümt die elektromobilität

voranzubringen

in wirklichkeit haben sie daran kein

interesse die rechnung für das

desinteresse der politik werden wir

möglicherweise in fünf oder in zehn

jahren alle bezahlen

denn dann besteht das große risiko dass

elektroautos in china gebaut werden

unsere arbeitsplätze dann dort sind und

wir dann traurig sind dass wir heute

nicht mehr getan haben

dabei sind in deutschland die grundlagen

für das elektromobile zeitalter längst

gelegt das beweist dieses fahrzeug der

sogenannte visio m

den haben forscher und studenten der

technischen universität münchen zusammen

mit bmw und daimler entwickelt derzeit

optimales elektroauto aus karbon und mit

160 kilometer reichweite wirklich

revolutionär daran ist der preis wir

haben von der tu münchen mit unseren

kosten modellen dieses fahrzeug

durchgerechnet und mit annahmen

das batterie preise noch mal deutlich

sinken werden bis zum jahr 2020 und wir

davon ausgehen wir gehen in große

stückzahlen könnte man so ein fahrzeug

für etwa 16.000 euro anbieten

nur 16.000 euro ein erschwinglicher

preis

doch von den deutschen autobauern hat

bisher keiner daran interesse wird der

wagen überhaupt je gebaut

wir zeigen das auto auch bei allen

entsprechenden partnern vor

unser fahrzeug ist bis asien hinweg

bekannt werden neulich partner hier die

das ganz genau im detail kannten

ich glaube fest dass das im nächsten

jahrzehnt in serie kommen wird ich kann

ihnen nicht versprechen ob es ein

deutscher hersteller dann wahrscheinlich

läuft es also doch wieder so deutsche

technik und das geschäft damit macht das

ausland und dann

geht es uns wie wir mit dem transrapid

da haben unsere erste chinesin gezeigt

dass der nicht nur im kreis sondern auch

geradeaus fahren kann und das darf uns
907
00:42:53,200 --> 00:00:00,000
nicht noch mal passieren

\end{document}
